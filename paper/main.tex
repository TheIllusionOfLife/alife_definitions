\documentclass[letterpaper]{article}
\usepackage{natbib,alifeconf}
\usepackage{graphicx,amsmath,amssymb,booktabs,xcolor,multirow}
\usepackage{url,hyperref}

\title{Life Definitions Disagree: An Empirical Benchmark\\of Competing Operationalizations in a Shared Digital Ecology}
\author{Anonymous}

\begin{document}
\maketitle

% ============================================================================
\begin{abstract}
Multiple operational definitions of life exist, yet they are rarely applied to
the same system under controlled conditions.  We present an empirical benchmark
comparing four definitions---textbook 7-criteria (D1), Darwinian/NASA (D2),
autonomy/organizational closure (D3), and information maintenance (D4)---by
scoring the same organism populations across five environment regimes in a
shared digital ecology.  Three co-existing family types with different ablated
capabilities serve as natural controls.  We report the disagreement structure
among definitions, identify regime-dependent divergences, and evaluate
predictive validity against population robustness.  Our benchmark suite and
adapter code are released as an open artifact for the ALife community.
\end{abstract}

Submission type: \textbf{Full Paper}

% ============================================================================
\section{Introduction}
\label{sec:intro}

% Why definitions matter (life detection, ALife foundations)
% Gap: most work adopts one definition
% Your approach: empirical comparison in the same world

\textbf{[TODO]} Motivate the benchmark: no prior work systematically compares
multiple definitions on the same digital organisms.

% ============================================================================
\section{Related Work}
\label{sec:related}

% Definitions of life (brief, functional)
% Benchmarking philosophies (not deep philosophy---keep it ALife)

\textbf{[TODO]} Brief survey of D1--D4 lineage and prior benchmarking efforts.

% ============================================================================
\section{System and Baseline}
\label{sec:system}

% Summarize your current system as substrate
% Cite that the 7-criteria integration exists and is validated

\textbf{[TODO]} Hybrid swarm-organism substrate, Mode~B with three family
profiles (F1~full, F2~Darwinian, F3~autonomy).

% ============================================================================
\section{Definition Adapters}
\label{sec:adapters}

% D1--D4 operationalizations
% Trace schema

\subsection{D1: Textbook 7-Criteria}
\textbf{[TODO]} Dynamic, degradation, coupling conditions per criterion;
geometric mean aggregate.

\subsection{D2: Darwinian / NASA}
\textbf{[TODO]} Sustained reproduction, heritability, differential success.

\subsection{D3: Autonomy / Organizational Closure}
\textbf{[TODO]} Transfer entropy influence graph, SCC closure score.

\subsection{D4: Information Maintenance}
\textbf{[TODO]} Genome causally predicts fitness, information preserved
across generations.

% ============================================================================
\section{Experiments}
\label{sec:experiments}

% Regimes E1--E5
% Held-out seeds
% Metrics and stats

\textbf{[TODO]} Five regimes (E1--E5), calibration seeds 0--99,
test seeds 100--199, statistical methods.

% ============================================================================
\section{Results}
\label{sec:results}

% Disagreement heatmaps
% Agreement matrix
% Predictive validity under perturbations
% Case studies

\textbf{[TODO]} Heatmaps, Cohen's $\kappa$, Spearman rank correlations,
case studies of diagnostic disagreements.

% ============================================================================
\section{Discussion}
\label{sec:discussion}

% What disagreements imply
% Limits: adapter dependence, substrate bias
% How others can plug in their systems

\textbf{[TODO]} Interpretation of disagreement structure, limitations,
extensibility of the benchmark.

% ============================================================================
\section{Artifact Release}
\label{sec:artifact}

% Benchmark suite structure + minimal reproduction instructions

\textbf{[TODO]} Repository structure, reproduction instructions, adapter API.

% ============================================================================
\section*{Acknowledgments}
\textbf{[TODO]}

\bibliographystyle{apalike}
\bibliography{references}

\end{document}
